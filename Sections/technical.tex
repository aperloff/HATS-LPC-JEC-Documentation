%!TEX root = ../2015_06_16-HATS-LPC-JEC.tex

\frame{
\frametitle{What form do the corrections come in? Text File}
\vspace*{0.3cm}
	%\begin{textblock}{14.5}(0.0,0.7)
		\begin{center}
%			\scalebox{.9}{
				%\documentclass[dvipsnames]{standalone}
\usepackage{color}
\usepackage{tikz}
\usetikzlibrary{arrows,shapes,shapes.multipart,backgrounds,calc,decorations.text,decorations.pathreplacing,matrix,shadings}
\tikzstyle{every picture}+=[remember picture]
\tikzstyle{na} = [baseline=-.5ex]

\begin{document}

\tikzstyle{levels} = [rectangle, draw, text width=6em, text centered, rounded corners, minimum height=2em, midway, shading=radial,outer color=Gray,middle color=white,inner color=white, Gray]
\tikzstyle{arrow} = [draw, -latex']
\tikzstyle{line} = [draw, -]

\newcommand*{\mytextstyle}{\sffamily\Large\bfseries\color{black!85}}
\newcommand{\myarrowstart}[9]{%
% inner radius, middle radius, outer radius, start angle,
% end angle, tip protusion angle, options, text
  \pgfmathsetmacro{\start}{#1}
  \pgfmathsetmacro{\rin}{#2}
  \pgfmathsetmacro{\rmid}{#3}
  \pgfmathsetmacro{\rout}{#4}
  \pgfmathsetmacro{\astart}{#5}
  \pgfmathsetmacro{\aend}{#6}
  \pgfmathsetmacro{\atip}{#7}
  \fill[#8] (\start+\astart,\rin) -- (\start+\aend,\rin) -- (\start+\aend+\atip,\rmid)
        -- (\start+\aend,\rout) -- (\start+\astart,\rout) -- (\start+\astart,\rmid)
        -- cycle;
  \node[font = \sffamily, align=center,text width=\aend-\astart,anchor = west, yshift=-0.0ex] (arrowend) at (\start+\astart,\rmid) {\mytextstyle{#9}};
%  \path[font = \sffamily, decoration = {text along path, text = {|\mytextstyle|#9},
%    text align = {align = center}, raise = -0.5ex}, decorate]
%    (\start+\astart+0.4*\atip,\rmid) -- (\start+\aend+0.4*\atip,\rmid);
}
\newcommand{\myarrow}[9]{%
% inner radius, middle radius, outer radius, start angle,
% end angle, tip protusion angle, options, text
  \pgfmathsetmacro{\start}{#1}
  \pgfmathsetmacro{\rin}{#2}
  \pgfmathsetmacro{\rmid}{#3}
  \pgfmathsetmacro{\rout}{#4}
  \pgfmathsetmacro{\astart}{#5}
  \pgfmathsetmacro{\aend}{#6}
  \pgfmathsetmacro{\atip}{#7}
  \fill[#8] (\start+\astart,\rin) -- (\start+\aend,\rin) -- (\start+\aend+\atip,\rmid) 
        -- (\start+\aend,\rout) -- (\start+\astart,\rout) -- (\start+\astart+\atip,\rmid)
        -- cycle;
  \path[font = \sffamily, decoration = {text along path, text = {|\mytextstyle|#9},
    text align = {align = center}, raise = -0.5ex}, decorate]
    (\start+\astart+0.75*\atip,\rmid) -- (\start+\aend+0.75*\atip,\rmid);
}
\newcommand{\myuphalfarrow}[9]{%
% inner radius, middle radius, outer radius, start angle,
% end angle, tip protusion angle, options, text
  \pgfmathsetmacro{\start}{#1}
  \pgfmathsetmacro{\rin}{#2}
  \pgfmathsetmacro{\rmid}{#3}
  \pgfmathsetmacro{\rout}{#4}
  \pgfmathsetmacro{\astart}{#5}
  \pgfmathsetmacro{\aend}{#6}
  \pgfmathsetmacro{\atip}{#7}
  \fill[#8] (\start+\astart,\rout) -- (\start+\aend,\rout) -- (\start+\aend+\atip,\rin) 
        -- (\start+\astart+\atip,\rin) -- cycle;
  \path[font = \sffamily, decoration = {text along path, text = {|\mytextstyle|#9},
    text align = {align = center}, raise = -0.5ex}, decorate]
    (\start+\astart+0.4*\atip,\rmid) -- (\start+\aend+0.4*\atip,\rmid);
}
\newcommand{\mylowhalfarrow}[9]{%
% inner radius, middle radius, outer radius, start angle,
% end angle, tip protusion angle, options, text
  \pgfmathsetmacro{\start}{#1}
  \pgfmathsetmacro{\rin}{#2}
  \pgfmathsetmacro{\rmid}{#3}
  \pgfmathsetmacro{\rout}{#4}
  \pgfmathsetmacro{\astart}{#5}
  \pgfmathsetmacro{\aend}{#6}
  \pgfmathsetmacro{\atip}{#7}
  \fill[#8] (\start+\astart+\atip,\rin) -- (\start+\aend+\atip,\rin) -- (\start+\aend,\rout)
        -- (\start+\astart,\rout) -- cycle;
  \path[font = \sffamily, decoration = {text along path, text = {|\mytextstyle|#9},
    text align = {align = center}, raise = -0.5ex}, decorate]
    (\start+\astart+0.4*\atip,\rmid) -- (\start+\aend+0.4*\atip,\rmid);
}
\newcommand{\myarrowend}[9]{%
% inner radius, middle radius, outer radius, start angle,
% end angle, tip protusion angle, options, text
  \pgfmathsetmacro{\start}{#1}
  \pgfmathsetmacro{\rin}{#2}
  \pgfmathsetmacro{\rmid}{#3}
  \pgfmathsetmacro{\rout}{#4}
  \pgfmathsetmacro{\astart}{#5}
  \pgfmathsetmacro{\aend}{#6}
  \pgfmathsetmacro{\atip}{#7}
  \fill[#8] (\start+\astart,\rin) -- (\start+\aend,\rin) -- (\start+\aend,\rmid)
        -- (\start+\aend,\rout) -- (\start+\astart,\rout) -- (\start+\astart+\atip,\rmid)
        -- cycle;
  \node[font = \sffamily, align=center,text width=\aend-\astart,anchor = west, yshift=-0.0ex] (arrowend) at (\start+\astart+0.75*\atip,\rmid) {\mytextstyle{#9}};
}

\begin{tikzpicture}[node distance = 2cm, auto, scale=0.8]
\scriptsize
    % Place nodes
    \myarrowstart{0}{0.5}{1.5}{2.5}{0}{1.75}{0.5}{Cyan,draw = Cyan, very thick}{\scriptsize{Reconstructed Jets}}

    \myuphalfarrow{1.95}{1.52}{1.75}{2.5}{0}{2.0}{0.5}{Purple,draw = Purple, very thick}{|\scriptsize|{MC + RC}}
    \mylowhalfarrow{1.95}{1.48}{1.}{0.5}{0}{2.0}{0.5}{Cyan,draw = Cyan, very thick}{|\scriptsize|{MC}}

    \node [font = \sffamily, align=left,text width=2.cm,anchor = west, yshift=-0.0ex] (l1data) at (2.4,2.2) {\mytextstyle\small\color{Black}\scriptsize{Pileup}};


    \myarrow{4.15}{0.5}{1.5}{2.5}{0}{2.95}{0.5}{Red!80,draw = Red!80, very thick}{|\scriptsize|{MC}}
    \node [font = \sffamily, align=left,text width=2.cm,anchor = west, yshift=-0.0ex] (l2l3) at (4.35,2.2) {\mytextstyle\small\color{Black}\scriptsize{Response $(p_{T},\eta)$ }};


    \myuphalfarrow{7.3}{1.52}{1.75}{2.5}{0}{1.9}{0.5}{BurntOrange,draw = BurntOrange, very thick}{|\scriptsize|{dijets}}
    \node [font = \sffamily, align=left,text width=2.cm,anchor = west, yshift=-0.0ex] (l2res) at (7.4,2.2) {\mytextstyle\small\color{Black}\scriptsize{Residuals$(\eta)$}};
    \mylowhalfarrow{7.3}{1.48}{1.25}{0.5}{0}{1.9}{0.5}{yellow!20,draw = yellow!20, very thick}{}


   \myuphalfarrow{9.4}{1.52}{1.75}{2.5}{0}{2.4}{0.5}{BurntOrange,draw = BurntOrange, very thick}{|\scriptsize|{   $\gamma$/Z$+$jet, MJB}}
    \node [font = \sffamily, align=left,text width=2.cm,anchor = west, yshift=-0.0ex] (l3res) at (9.6,2.2) {\mytextstyle\small\color{Black}\scriptsize{Residuals$(p_T)$}};
    \mylowhalfarrow{9.4}{1.48}{1.25}{0.5}{0}{2.4}{0.5}{yellow!20,draw = yellow!20, very thick}{}

 
   \myarrow{12.0}{0.5}{1.5}{2.5}{0}{1.}{0.5}{Yellow!80,draw = Yellow!80, very thick}{|\scriptsize|{MC}}
    \node [font = \sffamily, align=left,text width=2.cm,anchor = west, yshift=-0.0ex] (l5) at (12.,2.2) {\mytextstyle\small\color{Black}\scriptsize{ Flavor}};


    \myarrowend{13.20}{0.5}{1.5}{2.5}{0}{2.}{0.5}{Green!90,draw = Green!90, very thick}{\scriptsize{Calibrated Jets}}
    
    \node [font = \sffamily, align=left,text width=2.75cm,anchor = west, yshift=-0.0ex] (MC) at (2.25,0.) {\mytextstyle\small\color{RoyalBlue}Applied on MC};
    \node [font = \sffamily, align=left,text width=2.75cm,anchor = west, yshift=-0.0ex] (DATA) at (2.25,3.) {\mytextstyle\small\color{YellowOrange}Applied on data};

    \node [font = \sffamily, align=left,text width=5cm,anchor = west, yshift=-0.0ex] (INSITU) at (7.6,3.) %%{\mytextstyle\footnotesize{From in-situ MPF/Z-jet, flat in $p_{T}$}};
   {\mytextstyle\footnotesize{}};
    \node [font = \sffamily, align=left,text width=5cm,anchor = west, yshift=-0.0ex] (STARTBRACE) at (7.6,0.) {};
    \node [font = \sffamily, align=left,text width=5cm,anchor = west, yshift=-0.0ex] (ENDBRACE) at (12.65,0.5) {};
    \path[arrow,YellowOrange,thick, shorten <= -0.5cm] (DATA) -- (INSITU);
%    \draw [thick, decorate, decoration = {brace, amplitude = 10pt, mirror}, xshift = 0pt, yshift = 0pt] (7.25,0.75) -- (12.0,0.75) node [black, midway, xshift = 0cm, yshift = -0.8cm] {\mytextstyle\small{stuff}};
 %   \draw [thick, decorate, decoration = {brace, amplitude = 10pt, mirror}, xshift = 0pt, yshift = 0pt] (7.7,0.75) -- (12.65,0.75) node [levels, xshift = 0cm, yshift = -1.0cm] {\mytextstyle\scriptsize{L2L3Res}};

   \path[arrow,RoyalBlue,thick, shorten <= -0.5cm] (MC) -- (STARTBRACE);

 %   \node [levels, below of=MC, xshift = 3.6cm, yshift = 1.85cm, thick] (L1) {\mytextstyle\small{L1}};
 %   \node [levels, right of=L1, xshift = 4.4cm, yshift = -0.15cm, thick] (L2L3) {\mytextstyle\small{L2L3}};
\end{tikzpicture}

\end{document}

				\includegraphics[width=\textwidth]{images/txtfile.png}
%			}
		\end{center}
%	\end{textblock}
	
	\begin{block}{How to read the JEC text files?}
		\begin{itemize}
		\item The \textbf{top row} establishes the definitions for the parameters to follow, the correction factor and the correction level being applied.
		\item The \textbf{first two columns} are the $\eta$ bins.
		\item The next column tells you how many numbers will follow it (in this case 9).
		\item The next number is the lower bound of the $\rho$, $p_{T}$ and jet area bins (the upper bound is the third number after that).
		\item Intsructions on how to apply the Jet Energy Corrections from a text file in FWLite environment. \\ 
		\footnotesize
		\href{https://twiki.cern.ch/twiki/bin/view/CMSPublic/WorkBookJetEnergyCorrections}{https://twiki.cern.ch/twiki/bin/view/CMSPublic/WorkBookJetEnergyCorrections}
		\end{itemize}
	\end{block}
}